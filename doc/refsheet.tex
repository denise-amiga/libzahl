\documentclass[10pt,draft]{article}
\usepackage[margin=1in]{geometry}
\usepackage{amsmath, amssymb, mathtools}
\DeclarePairedDelimiter\ab{\lvert}{\rvert}

\newcommand{\size}{{\tt size\_t}}
\newcommand{\ullong}{{\tt unsigned long long int}}

\newcommand{\entry}[3]{ #2 & {\tt #1} & #3 \\ }
\newcommand{\entrycont}[1]{ & & \hspace*{2ex} #1 \\ }
\newcommand{\entryTwo}[4]{\entry{#1}{#2}{#3}\entrycont{#4}}
\newcommand{\entryThree}[5]{\entryTwo{#1}{#2}{#3}{#4}\entrycont{#5}}
\newcommand{\entryFour}[6]{\entryThree{#1}{#2}{#3}{#4}{#5}\entrycont{#6}}
\newcommand{\entryFive}[7]{\entryFour{#1}{#2}{#3}{#4}{#5}{#6}\entrycont{#7}}

\begin{document}



{\Huge libzahl}
\vspace{1ex}

Unless specified otherwise, returns are {\tt void} and all parameters are of type {\tt z\_t}.
\vspace{1.5em}



\hspace{-0.8em}
\begin{tabular}{lll}



\textbf{Initialisation} \\
\entryThree{zsetup(env)} {Initialise libzahl}   {must be called before any other function is}
                                                {used, {\tt env} is a {\tt jmp\_buf} all functions will}
                                                {{\tt longjmp} to --- with value 1 --- on error}
\entry     {zunsetup()}  {Deinitialise libzahl} {will free any pooled memory}
\entry     {zinit(a)}    {Initialise $a$}       {call once before use in any other function}
\entry     {zfree(a)}    {Deinitialise $a$}     {must not be used again before reinitialisation}
\\

\textbf{Error handling} \\
\entryTwo{zerror(a)}  {Get error code}          {returns {\tt enum zerror}, and stores}
                                                {description in {\tt const char **a}}
\entryTwo{zperror(a)} {Print error description} {behaves like {\tt perror(a)}, {\tt a} is a,}
                                                {possibly {\tt NULL} or $\varepsilon$, {\tt const char *}}
%\\

\textbf{Arithmetic} \\
\entry{zadd(a, b, c)}        {$a \gets b + c$}            {}
\entry{zsub(a, b, c)}        {$a \gets b - c$}            {}
\entry{zmul(a, b, c)}        {$a \gets b \cdot c$}        {}
\entry{zmodmul(a, b, c, d)}  {$a \gets b \cdot c \mod d$} {$0 \le a < \ab{d}$}
\entry{zdiv(a, b, c)}        {$a \gets b / c$}            {rounded towards zero}
\entry{zdivmod(a, b, c, d)}  {$a \gets c / d$}            {rounded towards zero}
\entry{zdivmod(a, b, c, d)}  {$b \gets c \mod d$}         {$0 \le b < \ab{d}$}
\entry{zmod(a, b, c)}        {$a \gets b \mod c$}         {$0 \le a < \ab{c}$}
%\entry{zdiv\_exact(a, b, c)} {$a \gets b / c$}            {assumes $c \vert d$}
\entry{zsqr(a, b)}           {$a \gets b^2$}              {}
\entry{zmodsqr(a, b, c)}     {$a \gets b^2 \mod c$}       {$0 \le a < \ab{c}$}
\entry{zsqr(a, b)}           {$a \gets b^2$}              {}
\entry{zpow(a, b, c)}        {$a \gets b^c$}              {}
\entry{zpowu(a, b, c)}       {$a \gets b^c$}              {{\tt c} is an \ullong{}}
\entry{zmodpow(a, b, c, d)}  {$a \gets b^c \mod d$}       {$0 \le a < \ab{d}$}
\entry{zmodpowu(a, b, c, d)} {$a \gets b^c \mod d$}       {ditto, {\tt c} is an \ullong{}}
\entry{zabs(a, b)}           {$a \gets \ab{b}$}           {}
\entry{zneg(a, b)}           {$a \gets -b$}               {}
\\

\textbf{Assignment} \\
\entry   {zset(a, b)}            {$a \gets b$}           {}
\entry   {zseti(a, b)}           {$a \gets b$}           {{\tt b} is an {\tt int64\_t}}
\entry   {zsetu(a, b)}           {$a \gets b$}           {{\tt b} is a {\tt uint64\_t}}
\entry   {zsets(a, b)}           {$a \gets b$}           {{\tt b} is a decimal {\tt const char *}}
%\entryTwo{zsets\_radix(a, b, c)} {$a \gets b$}           {{\tt b} is a radix $c$ {\tt const char *},}
%                                                         {{\tt c} is an \ullong{}}
\entry   {zswap(a, b)}           {$a \leftrightarrow b$} {}
\\

\textbf{Comparison} \\
\entry{zcmp(a, b)}    {Compare $a$ and $b$}           {returns {\tt int} $\mbox{sgn}(a - b)$}
\entry{zcmpi(a, b)}   {Compare $a$ and $b$}           {ditto, {\tt b} is an {\tt int64\_t}}
\entry{zcmpu(a, b)}   {Compare $a$ and $b$}           {ditto, {\tt b} is a {\tt uint64\_t}}
\entry{zcmpmag(a, b)} {Compare $\ab{a}$ and $\ab{b}$} {returns {\tt int} $\mbox{sgn}(\ab{a} - \ab{b})$}
\\



\end{tabular}
\newpage
\hspace{-0.8em}
\begin{tabular}{lll}



\textbf{Bit operations} \\
\entry{zand(a, b, c)}      {$a \gets b \wedge c$}         {bitwise}
\entry{zor(a, b, c)}       {$a \gets b \vee c$}           {bitwise}
\entry{zxor(a, b, c)}      {$a \gets b \oplus c$}         {bitwise}
\entry{znot(a, b, c)}      {$a \gets \lnot b$}            {bitwise, cut at highest set bit}
\entry{zlsh(a, b, c)}      {$a \gets b \cdot 2^c$}        {{\tt c} is a \size{}}
\entry{zrsh(a, b, c)}      {$a \gets b / 2^c$}            {ditto, rounded towards zero}
\entry{ztrunc(a, b, c)}    {$a \gets b \mod 2^c$}         {ditto, $a$ shares signum with $b$}
\entry{zbits(a)}           {Get number of used bits}      {returns \size{}, 1 if $a = 0$}
\entry{zlsb(a)}            {Get index of lowest set bit}  {returns \size{}, {\tt SIZE\_MAX} if $a = 0$}
\entry{zbtest(a, b)}       {Is bit $b$ in $a$ set?}       {{\tt b} is a \size{}, returns {\tt int}}
\entry{zbset(a, b, c, 1)}  {$a \gets b$, set bit $c$}     {{\tt c} is a \size{}}
\entry{zbset(a, b, c, 0)}  {$a \gets b$, clear bit $c$}   {ditto}
\entry{zbset(a, b, c, -1)} {$a \gets b$, flip bit $c$}    {ditto}
\entry{zsplit(a, b, c, d)} {$a \gets c / 2^d$}            {{\tt d} is a \size{}, rounded towards zero}
\entry{zsplit(a, b, c, d)} {$b \gets c \mod 2^d$}         {ditto, $b$ shares signum with $c$}
\\

\textbf{Conversion to string} \\
\entryFive{zstr(a, b, c)}           {Convert $a$ to decimal}   {returns the resulting {\tt const char *}}
                                                               {--- {\tt b} unless {\tt b} is
                                                                    {\tt NULL}, --- $c$ must be}
                                                               {either 0 or at least the length of the}
                                                               {resulting string but at most the}
                                                               {allocation size of {\tt b} minus 1}
%\entry    {zstr\_radix(a, b, c, d)} {Convert $a$ to radix $d$} {ditto, {\tt d} is an \ullong{}}
\entry    {zstr\_length(a, b)}      {Get string length of $a$} {returns \size{} length of $a$ in radix $b$}
\\

\textbf{Marshallisation} \\
\entryTwo{zsave(a, b)}    {Marshal $a$ into $b$}    {returns \size{} number of saved bytes,}
                                                    {{\tt b} is a {\tt void *\_t}}
\entry   {zsave(a, NULL)} {Get marshal-size of $a$} {returns \size{}}
\entryTwo{zload(a, b)}    {Unmarshal $a$ from $b$}  {returns \size{} number of read bytes,}
                                                    {{\tt b} is a {\tt const void *\_t}}
%\\

\textbf{Number theory} \\
\entry    {zsignum(a, b)}     {$a \gets \mbox{sgn} b$} {}
\entry    {zeven(a)}          {Is $a$ even?}           {returns {\tt int} 1 (true) or 0 (false)}
\entry    {zeven\_nonzero(a)} {Is $a$ even?}           {ditto, assumes $a \neq 0$}
\entry    {zodd(a)}           {Is $a$ odd?}            {returns {\tt int} 1 (true) or 0 (false)}
\entry    {zodd\_nonzero(a)}  {Is $a$ odd?}            {ditto, assumes $a \neq 0$}
\entry    {zzero(a)}          {Is $a$ zero?}           {returns {\tt int} 1 (true) or 0 (false)}
\entry    {zgcd(a, b, c)}     {$a \gets \gcd(c, b)$}   {$a < 0$ if $b < 0 \wedge c < 0$}
\entryFive{zptest(a, b, c)}   {Is $b$ a prime?}        {$c$ runs of Miller--Rabin, returns}
                                                       {{\tt enum zprimality} {\tt NONPRIME} (0)}
                                                       {(and stores the witness in {\tt a} unless}
                                                       {{\tt a} is {\tt NULL}), {\tt PROBABLY\_PRIME} (1), or}
                                                       {{\tt PRIME} (2)}
%\\

\textbf{Random numbers} \\
\entryTwo{zrand(a, b, UNIFORM, d)} {$a \xleftarrow{\$} \textbf{Z}_d$}
         {{\tt b} is a {\tt enum zranddev}, e.g.}
         {{\tt DEFAULT\_RANDOM}, {\tt FASTEST\_RANDOM}}
\\



\end{tabular}
\end{document}
