\chapter{Miscellaneous}
\label{chap:Miscellaneous}

In this chapter, we will learn some miscellaneous
functions. It might seem counterintuitive to start
with miscellanea, but it is probably a good idea
to read this before arithmetics and more advanced
topics. You may read \secref{sec:Marshalling}
later. Before reading this chapter you should
have read \chapref{chap:Get started}.


\vspace{1cm}
\minitoc


\newpage
\section{Assignment}
\label{sec:Assignment}

To be able to do anything useful, we must assign
values to integers. There are three functions for
this: {\tt zseti}, {\tt zsetu}, and {\tt zsets}.
The last letter in the names of these function
describe the data type of the input, `i', `u',
and `s' stand for `integer', `unsigned integer',
and `string`, respectively. These resemble the
rules for the format strings in the family of
{\tt printf}-functions. `Integer' of course refer
to `signed integer'; for integer types in C,
part from {\tt char}, the keyword {\tt signed}
is implicit.

Consider {\tt zseti},

\begin{alltt}
   \textcolor{c}{z_t two;}
   \textcolor{c}{zinit(two);}
   zseti(two, 2);
\end{alltt}

\noindent
assignes {\tt two} the value 2. The data type of
the second parameter of {\tt zseti} is {\tt int64\_t}.
It will accept any integer value in the range
$[-2^{63},~2^{63} - 1] = [-9223372036854775808,~9223372036854775807]$,
independently of the machine.\footnote{{\tt int64\_t}
is defined to be a signed 64-bit integer using two's
complement representation.} If this range so not wide
enough, it may be possible to use {\tt zsetu}. Its
second parameter of the type {\tt uint64\_t}, and thus
its range is $[0,~2^{64} - 1] = [0,~18446744073709551615]$.
If a need negative value is desired, {\tt zsetu} can be
combined with {\tt zneg} \psecref{sec:Sign manipulation}.

For enormous constants or textual input, {\tt zsets}
can be used. {\tt zsets} will accept any numerical
value encoded in decimal ASCII, that only contain
digits, \emph{not} decimal points, whitespace,
apostrophes, et cetera. However, an optional plus
sign or, for negative numbers, an ASCII minus sign
may be used as the very first character. Note that
a proper UCS minus sign is not supported.

Using what we have learned so far, and {\tt zstr}
which we will learn about in \secref{sec:String output},
we can construct a simple program that calculates the
sum of a set of number.

\begin{alltt}
   \textcolor{c}{#include <stdio.h>
   #include <stdlib.h>
   #include <zahl.h>}

   int
   main(int argc, char *argv[]) \{
       z_t sum, temp;
       \textcolor{c}{jmp_buf failenv;
       char *sbuf, *argv0 = *argv;
       if (setjmp(failenv)) \{
           zperror(argv0);
           return 1;
       \}
       zsetup(failenv);
       zinit(sum);
       zinit(term);}
       zsetu(sum, 0);
       for (argv++; *argv; argv++) \{
           zsets(term, *argv);
           zadd(sum, sum, term);
       \}
       \textcolor{c}{printf("\%s\textbackslash{}n", (sbuf = zstr(sum, NULL, 0)));
       free(sbuf);
       zfree(sum);
       zfree(term);
       zunsetup();
       return 0;}
   \}
\end{alltt}

Another form of assignment available in libzahl is
copy-assignment. This done using {\tt zset}. As
easily observable, {\tt zset} is named like
{\tt zseti}, {\tt zsetu}, and {\tt zsetu}, but
without the input-type suffix. The lack of a
input-type suffix means that the input type is
{\tt z\_t}. {\tt zset} copies value of second
parameter into the reference in the first. For
example, if {\tt v}, of the type {\tt z\_t}, has
value 10, then {\tt a} will too after the instruction

\begin{alltt}
   zset(a, v);
\end{alltt}

{\tt zset} does not necessarily make an exact
copy of the input. If, in the example above, the
{\tt a->alloced} is greater than or equal to
{\tt v->used}, {\tt a->alloced} and {\tt a->chars}
are preserved, of course, the content of
{\tt a->chars} is overridden. If however,
{\tt a->alloced} is less then {\tt v->used},
{\tt a->alloced} is assigned a minimal value at
least as great as {\tt v->used} that is a power
of 2, and {\tt a->chars} is updated accordingly
as described in \secref{sec:Integer structure}.
This of course does not apply if {\tt v} has the
value 0; in such cases {\tt a->sign} is simply
set to 0.

{\tt zset}, {\tt zseti}, {\tt zsetu}, and
{\tt zsets} require that the output-parameter
has been initialised with {\tt zinit} or an
equally acceptable method as described in
\secref{sec:Create an integer}.

{\tt zset} is often unnecessary, of course
there are cases where it is needed. In some case
{\tt zswap} is enough, and advantageous.
{\tt zswap} is defined as

\begin{alltt}
   \textcolor{c}{static inline} void
   zswap(z_t a, z_t b)
   \{
       z_t t;
       *t = *a;
       *a = *b;
       *b = *t;
   \}
\end{alltt}

\noindent
however its implementation is optimised to be
around three times as fast. It just swaps the members
of the parameters, and thereby the values, There
is no rewriting of {\tt .chars} involved; thus
it runs in constant time. It also does not
require that any argument has be initialised.
After the call, {\tt a} will be initialised
if and only if {\tt b} was initialised, and
vice versa.


\newpage
\section{String output}
\label{sec:String output}

Few useful things can be done without creating
textual output of calculations. To convert a
{\tt z\_t} to ASCII string in decimal, we use the
function {\tt zstr}, declared as

\begin{alltt}
   char *zstr(z_t a, char *buf, size_t n);
\end{alltt}

\noindent
{\tt zstr} will store the string it creates into
{\tt buf} and return {\tt buf}. However, if {\tt buf}
is {\tt NULL}, a new memory segment is allocated
and returned. {\tt n} should be at least the length
of the resulting string sans NUL termiantion, but
not larger than the allocation size of {\tt buf}
minus 1 byte for NUL termiantion. If {\tt buf} is
{\tt NULL}, {\tt n} may be 0. However if {\tt buf}
is not {\tt NULL}, it is unsafe to let {\tt n} be
0, unless {\tt buf} has been allocated by {\tt zstr}
for a value of {\tt a} at least as larger as the
value of {\tt a} in the new call to {\tt zstr}.
Combining non-\texttt{NULL} {\tt buf} with 0 {\tt n}
is unsafe because {\tt zstr} will use a very fast
formula for calculating a value that is at least
as large as the resulting output length, rather
than the exact length.

The length of the string output by {\tt zstr} can
be predicted by {\tt zstr\_length}, decleared as

\begin{alltt}
   size_t zstr_length(z_t a, unsigned long long int radix);
\end{alltt}

\noindent
It will calculated the length of {\tt a} represented
in radix {\tt radix}, sans NUL termination. If
{\tt radix} is 10, the length for a decimal
representation is calculated.

Sometimes it is possible to never allocate a {\tt buf}
for {\tt zstr}. For example, in an implementation
of {\tt factor}, you can reuse the string of the
value to factorise, since all of its factors are
guaranteed to be no longer than the factored value.

\begin{alltt}
   void
   factor(char *value)
   \{
       size_t n = strlen(value);
       z_t product, factor;
       zsets(product, value);
       printf("\%s:", value);
       while (next_factor(product, factor))
           printf(" \%s", zstr(factor, value, n));
       printf("\verb|\|n");
   \}
\end{alltt}

Other times it is possible to allocate just
once, for example of creating a sorted output.
In such cases, the allocation can be done almost
transparently.

\begin{alltt}
   void
   output_presorted_decending(z_t *list, size_t n)
   \{
       char *buf = NULL;
       while (n--)
           printf("\%s\verb|\|n", (buf = zstr(*list++, buf, 0)));
   \}
\end{alltt}

\noindent
Note, this example assumes that all values are
non-negative.



\newpage
\section{Comparison}
\label{sec:Comparison}

libzahl defines four functions for comparing
integers: {\tt zcmp}, {\tt zcmpi}, {\tt zcmpu},
and {\tt zcmpmag}. These follow the same naming
convention as {\tt zset}, {\tt zseti}, and
{\tt zsetu}, as described in \secref{sec:Assignment}.
{\tt zcmpmag} compares the absolute value, the
magnitude, rather than the proper value. These
functions are declared as

\begin{alltt}
   int zcmp(z_t a, z_t b);
   int zcmpi(z_t a, int64_t b);
   int zcmpu(z_t a, uint64_t b);
   int zcmpmag(z_t a, z_t b);
\end{alltt}

\noindent
They behave similar to {\tt memcmp} and
{\tt strcmp}.\footnote{And {\tt wmemcmp} and
{\tt wcscmp} if you are into that mess.}
The return value is defined

\vspace{1em}
\(
    \mbox{sgn}(a - b) =
    \left \lbrace \begin{array}{rl}
        -1 & \quad \textrm{if}~a < b \\
         0 & \quad \textrm{if}~a = b \\
        +1 & \quad \textrm{if}~a > b
    \end{array} \right .
\)
\vspace{1em}

\noindent
for {\tt zcmp}, {\tt zcmpi}, and {\tt zcmpu}.
The return for {\tt zcmpmag} value is defined

\vspace{1em}
\(
    \mbox{sgn}(\lvert a \rvert - \lvert b \rvert) =
    \left \lbrace \begin{array}{rl}
        -1 & \quad \textrm{if}~\lvert a \rvert < \lvert b \rvert \\
         0 & \quad \textrm{if}~\lvert a \rvert = \lvert b \rvert \\
        +1 & \quad \textrm{if}~\lvert a \rvert > \lvert b \rvert
    \end{array} \right .
\)
\vspace{1em}

\noindent
It is discouraged, stylistically, to compare
against, $-1$ and $+1$, rather, you should
always compare against $0$. Think of it as
returning $a - b$, or $\lvert a \rvert - \lvert b \rvert$
in the case of {\tt zcmpmag}.


\newpage
\section{Marshalling}
\label{sec:Marshalling}

libzahl is designed to provide efficient communication
for multi-processes applications, including running on
multiple nodes on a cluster computer. However, these
facilities require that it is known that all processes
run the same version of libzahl, and run on compatible
microarchitectures, that is, the processors must have
endianness, and the intrinsic integer types in C must
have the same widths on all processors. When this is not
the case, string conversion (see \secref{sec:Assignment}
and \secref{sec:String output}), but when it is the case
{\tt zsave} and {\tt zload} can be used. {\tt zsave} and
{\tt zload} are declared as

\begin{alltt}
   size_t zsave(z_t a, char *buf);
   size_t zload(z_t a, const char *buf);
\end{alltt}

\noindent
{\tt zsave} stores a version- and microarchitecture-depend
binary representation of {\tt a} in {\tt buf}, and returns
the number of bytes written to {\tt buf}. If {\tt buf} is
{\tt NULL}, the numbers that will be written is returned.
{\tt zload} unmarshals an integers from {\tt buf}, created
with {\tt zsave}, into {\tt a}, and returns the number of
read bytes. {\tt zload} and will return the value returned
by {\tt zsave}.
