\chapter{Arithmetic}
\label{chap:Arithmetic}

In this chapter, we will learn how to perform basic
arithmetic with libzahl: addition, subtraction,
multiplication, division, modulus, exponentiation,
and sign manipulation. \secref{sec:Division} is of
special importance.

\vspace{1cm}
\minitoc


\newpage
\section{Addition}
\label{sec:Addition}

To calculate the sum of two terms, we perform
addition using {\tt zadd}.

\vspace{1em}
$r \gets a + b$
\vspace{1em}

\noindent
is written as

\begin{alltt}
   zadd(r, a, b);
\end{alltt}

libzahl also provides {\tt zadd\_unsigned} which
has slightly lower overhead. The calculates the
sum of the absolute values of two integers.

\vspace{1em}
$r \gets \lvert a \rvert + \lvert b \rvert$
\vspace{1em}

\noindent
is written as

\begin{alltt}
   zadd_unsigned(r, a, b);
\end{alltt}

\noindent
{\tt zadd\_unsigned} has lower overhead than
{\tt zadd} because it does not need to inspect
or change the sign of the input, the low-level
function that performs the addition inherently
calculates the sum of the absolute values of
the input.

In libzahl, addition is implemented using a
technique called ripple-carry. It is derived
from that observation that

\vspace{1em}
$f : \textbf{Z}_n, \textbf{Z}_n \rightarrow \textbf{Z}_n$
\\ \indent
$f : a, b \mapsto a + b + 1$
\vspace{1em}

\noindent
only wraps at most once, that is, the
carry cannot exceed 1. CPU:s provide an
instruction specifically for performing
addition with ripple-carry over multiple words,
adds twos numbers plus the carry from the
last addition. libzahl uses assembly to
implement this efficiently. If however, an
assembly implementation is not available for
the on which machine it is running, libzahl
implements ripple-carry less efficiently
using compiler extensions that check for
overflow. In the event that neither an
assembly implementation is available nor
the compiler is known to support this
extension, it is implemented using inefficient
pure C code. This last resort manually
predicts whether an addition will overflow;
this could be made more efficent, but never
using the highest bit, in each character,
except to detect overflow. This optimisation
is however not implemented because it is
not deemed important enough and would
be detrimental to libzahl's simplicity.

{\tt zadd} and {\tt zadd\_unsigned} support
in-place operation:

\begin{alltt}
   zadd(a, a, b);
   zadd(b, a, b);           \textcolor{c}{/* \textrm{should be avoided} */}
   zadd_unsigned(a, a, b);
   zadd_unsigned(b, a, b);  \textcolor{c}{/* \textrm{should be avoided} */}
\end{alltt}

\noindent
Use this whenever possible, it will improve
your performance, as it will involve less
CPU instructions for each character-addition
and it may be possible to eliminate some
character-additions.


\newpage
\section{Subtraction}
\label{sec:Subtraction}

TODO % zsub zsub_unsigned


\newpage
\section{Multiplication}
\label{sec:Multiplication}

TODO % zmul zmodmul


\newpage
\section{Division}
\label{sec:Division}

To calculate the quotient or modulus of two integers,
use either of

\begin{alltt}
   void zdiv(z_t quotient, z_t dividend, z_t divisor);
   void zmod(z_t remainder, z_t dividend, z_t divisor);
   void zdivmod(z_t quotient, z_t remainder,
                z_t dividend, z_t divisor);
\end{alltt}

\noindent
These function \emph{do not} allow {\tt NULL}
for the output parameters: {\tt quotient} and
{\tt remainder}. The quotient and remainder are
calculated simultaneously and indivisibly, hence
{\tt zdivmod} is provided to calculated both, if
you are only interested in the quotient or only
interested in the remainder, use {\tt zdiv} or
{\tt zmod}, respectively.

These functions calculate a truncated quotient.
That is, the result is rounded towards zero. This
means for example that if the quotient is in
$(-1,~1)$, {\tt quotient} gets 0. That that, this
would not be the case for one of the sides of zero.
For example, if the quotient would have been
floored, negative quotients would have been rounded
away from zero. libzahl only provides truncated
division,

The remain is defined such that $n = qd + r$ after
calling {\tt zdivmod(q, r, n, d)}. There is no
difference in the remainer between {\tt zdivmod}
and {\tt zmod}. The sign of {\tt d} has no affect
on {\tt r}, {\tt r} will always, unless it is zero,
have the same sign as {\tt n}.

There are of course other ways to define integer
division (that is, \textbf{Z} being the codomain)
than as truncated division. For example integer
divison in Python is floored — yes, you did just
read `integer divison in Python is floored,' and
you are correct, that is not the case in for
example C. Users that want another definition
for division than truncated division are required
to implement that themselves. We will however
lend you a hand.

\begin{alltt}
   #define isneg(x) (zsignum(x) < 0)
   static z_t one;
   \textcolor{c}{__attribute__((constructor)) static}
   void init(void) \{ zinit(one), zseti(one, 1); \}

   static int
   cmpmag_2a_b(z_t a, z_b b)
   \{
       int r;
       zadd(a, a, a), r = zcmpmag(a, b), zrsh(a, a, 1);
       return r;
   \}
\end{alltt}

% Floored division
\begin{alltt}
   void \textcolor{c}{/* \textrm{All arguments most be unique.} */}
   divmod_floor(z_t q, z_t r, z_t n, z_t d)
   \{
       zdivmod(q, r, n, d);
       if (!zzero(r) && isneg(n) != isneg(d))
           zsub(q, q, one), zadd(r, r, d);
   \}
\end{alltt}

% Ceiled division
\begin{alltt}
   void \textcolor{c}{/* \textrm{All arguments most be unique.} */}
   divmod_ceiling(z_t q, z_t r, z_t n, z_t d)
   \{
       zdivmod(q, r, n, d);
       if (!zzero(r) && isneg(n) == isneg(d))
           zadd(q, q, one), zsub(r, r, d);
   \}
\end{alltt}

% Division with round half aways from zero
% This rounding method is also called:
%    round half toward infinity
%    commercial rounding
\begin{alltt}
   /* \textrm{This is how we normally round numbers.} */
   void \textcolor{c}{/* \textrm{All arguments most be unique.} */}
   divmod_half_from_zero(z_t q, z_t r, z_t n, z_t d)
   \{
       zdivmod(q, r, n, d);
       if (!zzero(r) && cmpmag_2a_b(r, d) >= 0) \{
           if (isneg(n) == isneg(d))
               zadd(q, q, one), zsub(r, r, d);
           else
               zsub(q, q, one), zadd(r, r, d);
       \}
   \}
\end{alltt}

\noindent
Now to the weird ones that will more often than
not award you a face-slap. % Had positive punishment
% been legal or even mildly pedagogical. But I would
% not put it past Coca-Cola.

% Division with round half toward zero
% This rounding method is also called:
%     round half away from infinity
\begin{alltt}
   void \textcolor{c}{/* \textrm{All arguments most be unique.} */}
   divmod_half_to_zero(z_t q, z_t r, z_t n, z_t d)
   \{
       zdivmod(q, r, n, d);
       if (!zzero(r) && cmpmag_2a_b(r, d) > 0) \{
           if (isneg(n) == isneg(d))
               zadd(q, q, one), zsub(r, r, d);
           else
               zsub(q, q, one), zadd(r, r, d);
       \}
   \}
\end{alltt}

% Division with round half up
% This rounding method is also called:
%     round half towards positive infinity
\begin{alltt}
   void \textcolor{c}{/* \textrm{All arguments most be unique.} */}
   divmod_half_up(z_t q, z_t r, z_t n, z_t d)
   \{
       int cmp;
       zdivmod(q, r, n, d);
       if (!zzero(r) && (cmp = cmpmag_2a_b(r, d)) >= 0) \{
           if (isneg(n) == isneg(d))
               zadd(q, q, one), zsub(r, r, d);
           else if (cmp)
               zsub(q, q, one), zadd(r, r, d);
       \}
   \}
\end{alltt}

% Division with round half down
% This rounding method is also called:
%     round half towards negative infinity
\begin{alltt}
   void \textcolor{c}{/* \textrm{All arguments most be unique.} */}
   divmod_half_down(z_t q, z_t r, z_t n, z_t d)
   \{
       int cmp;
       zdivmod(q, r, n, d);
       if (!zzero(r) && (cmp = cmpmag_2a_b(r, d)) >= 0) \{
           if (isneg(n) != isneg(d))
               zsub(q, q, one), zadd(r, r, d);
           else if (cmp)
               zadd(q, q, one), zsub(r, r, d);
       \}
   \}
\end{alltt}

% Division with round half to even
% This rounding method is also called:
%     unbiased rounding        (really stupid name)
%     convergent rounding      (also quite stupid name)
%     statistician's rounding
%     Dutch rounding
%     Gaussian rounding
%     odd–even rounding
%     bankers' rounding
% It is the default rounding method used in IEEE 754.
\begin{alltt}
   void \textcolor{c}{/* \textrm{All arguments most be unique.} */}
   divmod_half_to_even(z_t q, z_t r, z_t n, z_t d)
   \{
       int cmp;
       zdivmod(q, r, n, d);
       if (!zzero(r) && (cmp = cmpmag_2a_b(r, d)) >= 0) \{
           if (cmp || zodd(q)) \{
               if (isneg(n) != isneg(d))
                   zsub(q, q, one), zadd(r, r, d);
               else
                   zadd(q, q, one), zsub(r, r, d);
           \}
       \}
   \}
\end{alltt}

% Division with round half to odd
\newpage
\begin{alltt}
   void \textcolor{c}{/* \textrm{All arguments most be unique.} */}
   divmod_half_to_odd(z_t q, z_t r, z_t n, z_t d)
   \{
       int cmp;
       zdivmod(q, r, n, d);
       if (!zzero(r) && (cmp = cmpmag_2a_b(r, d)) >= 0) \{
           if (cmp || zeven(q)) \{
               if (isneg(n) != isneg(d))
                   zsub(q, q, one), zadd(r, r, d);
               else
                   zadd(q, q, one), zsub(r, r, d);
           \}
       \}
   \}
\end{alltt}

% Other standard methods include stochastic rounding
% and round half alternatingly, and what is is
% New Zealand called “Swedish rounding”, which is
% no longer used in Sweden, and is just normal round
% half aways from zero but with 0.5 rather than
% 1 as the integral unit, and is just a special case
% of a more general rounding method.

Currently, libzahl uses an almost trivial division
algorithm. It operates on positive numbers. It begins
by left-shifting the divisor as must as possible with
letting it exceed the dividend. Then, it subtracts
the shifted divisor from the dividend and add 1,
left-shifted as much as the divisor, to the quotient.
The quotient begins at 0. It then right-shifts
the shifted divisor as little as possible until
it no longer exceeds the diminished dividend and
marks the shift in the quotient. This process is
repeated on till the unshifted divisor is greater
than the diminished dividend. The final diminished
dividend is the remainder.



\newpage
\section{Exponentiation}
\label{sec:Exponentiation}

Exponentiation refers to raising a number to
a power. libzahl provides two functions for
regular exponentiation, and two functions for
modular exponentiation. libzahl also provides
a function for raising a number to the second
power, see \secref{sec:Multiplication} for
more details on this. The functions for regular
exponentiation are

\begin{alltt}
   void zpow(z_t power, z_t base, z_t exponent);
   void zpowu(z_t, z_t, unsigned long long int);
\end{alltt}

\noindent
They are identical, except {\tt zpowu} expects
and intrinsic type as the exponent. Both functions
calculate

\vspace{1em}
$power \gets base^{exponent}$
\vspace{1em}

\noindent
The functions for modular exponentiation are
\begin{alltt}
   void zmodpow(z_t, z_t, z_t, z_t modulator);
   void zmodpowu(z_t, z_t, unsigned long long int, z_t);
\end{alltt}

\noindent
They are identical, except {\tt zmodpowu} expects
and intrinsic type as the exponent. Both functions
calculate

\vspace{1em}
$power \gets base^{exponent} \mod modulator$
\vspace{1em}

The sign of {\tt modulator} does not affect the
result, {\tt power} will be negative if and only
if {\tt base} is negative and {\tt exponent} is
odd, that is, under the same circumstances as for
{\tt zpow} and {\tt zpowu}.

These four functions are implemented using
exponentiation by squaring. {\tt zmodpow} and
{\tt zmodpowu} are optimised, they modulate
results for multiplication and squaring at
every multiplication and squaring, rather than
modulating every at the end. Exponentiation
by modulation is a very simple algorithm which
can be expressed as a simple formula

\vspace{-1em}
\[ \hspace*{-0.4cm}
    a^b =
    \prod_{k \in \textbf{Z}_{+} ~:~ \left \lfloor \frac{b}{2^k} \hspace*{-1ex} \mod 2 \right \rfloor = 1}
    a^{2^k}
\]

\noindent
This is a natural extension to the
observations\footnote{The first of course being
that any non-negative number can be expressed
with the binary positional system. The latter
should be fairly self-explanatory.}

\vspace{-1em}
\[ \hspace*{-0.4cm}
    \forall b \in \textbf{Z}_{+} \exists B \subset \textbf{Z}_{+} : b = \sum_{i \in B} 2^i
    ~~~~ \textrm{and} ~~~~
    a^{\sum x} = \prod a^x.
\]

\noindent
The algorithm can be expressed in psuedocode as

\vspace{1em}
\hspace{-2.8ex}
\begin{minipage}{\linewidth}
\begin{algorithmic}
    \STATE $r, f \gets 1, a$
    \WHILE{$b \neq 0$}
      \STATE $r \gets r \cdot f$ {\bf unless} $2 \vert b$
      \STATE $f \gets f^2$ \textcolor{c}{\{$f \gets f \cdot f$\}}
      \STATE $b \gets \lfloor b / 2 \rfloor$
    \ENDWHILE
    \RETURN $r$ 
\end{algorithmic}
\end{minipage}
\vspace{1em}

\noindent
Modular exponentiation ($a^b \mod m$) by squaring can be
expressed as

\vspace{1em}
\hspace{-2.8ex}
\begin{minipage}{\linewidth}
\begin{algorithmic}
    \STATE $r, f \gets 1, a$
    \WHILE{$b \neq 0$}
      \STATE $r \gets r \cdot f \hspace*{-1ex}~ \mod m$ \textbf{unless} $2 \vert b$
      \STATE $f \gets f^2 \hspace*{-1ex}~ \mod m$
      \STATE $b \gets \lfloor b / 2 \rfloor$
    \ENDWHILE
    \RETURN $r$ 
\end{algorithmic}
\end{minipage}
\vspace{1em}

{\tt zmodpow} does \emph{not} calculate the
modular inverse if the exponent is negative,
rather, you should expect the result to be
1 and 0 depending of whether the base is 1
or not 1.


\newpage
\section{Sign manipulation}
\label{sec:Sign manipulation}

libzahl provides two functions for manipulating
the sign of integers:

\begin{alltt}
   void zabs(z_t r, z_t a);
   void zneg(z_t r, z_t a);
\end{alltt}

{\tt zabs} stores the absolute value of {\tt a}
in {\tt r}, that is, it creates a copy of
{\tt a} to {\tt r}, unless {\tt a} and {\tt r}
are the same reference, and then removes its sign;
if the value is negative, it becomes positive.

\vspace{1em}
\(
    r \gets \lvert a \rvert =
    \left \lbrace \begin{array}{rl}
        -a & \quad \textrm{if}~a \le 0 \\
        +a & \quad \textrm{if}~a \ge 0 \\
    \end{array} \right .
\)
\vspace{1em}

{\tt zneg} stores the negated of {\tt a}
in {\tt r}, that is, it creates a copy of
{\tt a} to {\tt r}, unless {\tt a} and {\tt r}
are the same reference, and then flips sign;
if the value is negative, it becomes positive,
if the value is positive, it becomes negative.

\vspace{1em}
\(
    r \gets -a
\)
\vspace{1em}

Note that there is no function for

\vspace{1em}
\(
    r \gets -\lvert a \rvert =
    \left \lbrace \begin{array}{rl}
         a & \quad \textrm{if}~a \le 0 \\
        -a & \quad \textrm{if}~a \ge 0 \\
    \end{array} \right .
\)
\vspace{1em}

\noindent
calling {\tt zabs} followed by {\tt zneg}
should be sufficient for most users:

\begin{alltt}
   #define my_negabs(r, a)  (zabs(r, a), zneg(r, r))
\end{alltt}
