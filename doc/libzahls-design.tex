\chapter{libzahl's design}
\label{chap:libzahl's design}

In this chapter, the design of libzahl is discussed.

\vspace{1cm}
\minitoc


\newpage
\section{Memory pool}
\label{sec:Memory pool}

Allocating memory dynamically is an expensive operation.
To improve performance, libzahl never deallocates memory
before the library is uninitialised, instead it pools
memory, that is no longer needed, for reuse.

Because of the memory pooling, this is a pattern to the
allocation sizes. In an allocation, a power of two
elements, plus a few elements that are discussed in
\secref{sec:Integer structure}, are allocated. That is,
the number multiplied by the size of an element.
Powers of two (growth factor 2) is not the most memory
efficient way to do this, but it is the simplest and
performance efficient. This power of two (sans the few
extra elements) is used to calculate — getting the index
of the only set bit — the index of the bucket in
which the allocation is stored when pooled. The buckets
are dynamic arrays with the growth factor 1.5. The
growth factor 1.5 is often used for dynamic arrays, it
is a good compromise between memory usage and performance.

libzahl also avoids allocating memory by having a set
of temporary variables predefined.


\newpage
\section{Error handling}
\label{sec:Error handling}

In C, it is traditional to return a sentiel value
in case an error has occurred, and set the value
of a global variable to describe the error that
has occurred. The programmer can choose whether to
check for errors, ignore errors where it does not
matter, or simple ignore errors altogether and let
the program eventually crash. This is a simple
technique that gives the programmer a better
understanding of what can happen. A great advantage
C has over most programming languages.

Another technique is to use long jumps on error.
This technique is not too common, but is has one
significant advantage. Error-checks need only be
preformed where the error can first be detected.
There is no need to check the return value at every
function return. This leads to cleaner code, if
there are many functions that can raise exceptional
conditions, and greater performance under some
conditions. This is why this technique is sometimes
used in high-performance libraries. libzahl uses
this technique.

Rather than writing

\begin{alltt}
   if (zadd(a, b, c))
       goto out;
\end{alltt}

\noindent
or a bit cleaner, if there are a lot of calls,

\begin{alltt}
   #define TRY(...) do if (__VA_ARGS__) goto out; while (0)
   \textcolor{c}{/* \textrm{\ldots} */}
   TRY(zadd(a, b, c));
\end{alltt}

\noindent
we write

\begin{alltt}
   jmp_buf env;
   if (setjmp(env))
       goto out;
   zsetup(env);
   \textcolor{c}{/* \textrm{\ldots} */}
   zadd(a, b, c);
\end{alltt}

You only need to call {\tt setjmp} and {\tt zsetup}
once, but can update the return point by calling
them once more.

If you don't need to check for errors, you can
disable error detection at compile-time. By defining
the {\tt ZAHL\_UNSAFE} C preprocessor definition
when compiling libzahl, and when compiling your
software that uses libzahl.


\newpage
\section{Integer structure}
\label{sec:Integer structure}

The data type used to represent a big integer with
libzahl is {\tt z\_t},\footnote{This name actually
violates the naming convention; it should be {\tt Z},
or {\tt Zahl} to avoid single-letter names. But this
violation is common-place.} defined as

\begin{alltt}
   typedef struct zahl z_t[1];
\end{alltt}

\noindent
where {\tt struct zahl} is defined as

\begin{alltt}
   struct zahl \{
       int sign;            \textcolor{c}{/* \textrm{\emph{not} short for `signum'} */}
       size_t used;
       size_t alloced;      \textcolor{c}{/* \textrm{short for `allocated'} */}
       zahl_char_t *chars;  \textcolor{c}{/* \textrm{short for `characters'} */}
   \};
\end{alltt}

\noindent
where {\tt zahl\_char\_t} is defined as

\begin{alltt}
   typedef uint64_t zahl_char_t;
\end{alltt}

\noindent
As a user, try not to think about anything else than

\begin{alltt}
   typedef \textcolor{c}{/* \textrm{ignore what is here} */} z_t[1];
\end{alltt}

\noindent
details can change in future versions of libzahl.

{\tt z\_t} is defined as a single-element array. This
is often called a reference, or a call-by-reference.
There are some flexibility issues with this, why
{\tt struct zahl} has beed added, but for most uses
with big integers, it makes things simpler. Particularly,
you need not work prepend {\tt \&} to variable when making
function calls, but the existence of {\tt struct zahl}
allows you do so if you so choose.

The {\tt .sign} member, is either $-1$, 0, or 1,
when the integer is negative, zero, or positive,
respectively. Whenever, {\tt .sign} is 0, the value
of {\tt .used} and {\tt .chars} are undefined.

{\tt .used} holds to the number of elements used in
{\tt .chars}, and {\tt .alloced} holds the allocation
side of {\tt .chars} measured in elements minus a few
extra elements that are always added to the allocation.
{\tt .chars} is a little-endian array of 64-bit digits,
these 64-bit digits are called `characters' in libzahl.
{\tt .chars} holds the absolute value of the
represented value.

Unless {\tt .sign} is 0, {\tt .chars} always contains
four extra elements, refered to as fluff. These are
merely allocated so functions can assume that they can
always manipulate groups of four characters, and need
not care about cases where the number of characters is
not a multiple of four. There are of course a few cases
when the precise number of characters is important.


\newpage
\section{Parameters}
\label{sec:Parameters}

The general order of parameters in libzahl functions
are: output integers, input integers, input data,
output data, parametric values. For example, in
addition, the out parameter is the first parameter.
But for marshalling and unmarshalling the buffer
is last. For random number generation the order is:
output, device, distribution, distribution parameters.
Whilst the distribution parameters are big integers,
they are not considered input integers. The order
of the input parameters are that of the order you
would write them using mathematical notation, this
also holds true if you include the output parameter
(as long as there is exactly one output,) for example

\vspace{1ex}
$a \gets b^c \mod d$
\vspace{1ex}

\noindent
is written

\begin{verbatim}
   zmodpow(a, b, c, d);
\end{verbatim}

\noindent
or

\begin{verbatim}
   zmodpowu(a, b, c, d);
\end{verbatim}

Like any self respecting bignum library, libzahl
supports using the same big integer reference as
for output as input, as long as all the output
parameters are mutually unique. For example

\begin{alltt}
   a += b;
\end{alltt}

\noindent
or

\begin{alltt}
   a = a + b;
\end{alltt}

\noindent
is written, using libzahl, as

\begin{alltt}
   zadd(a, a, b);
\end{alltt}

For commutative functions, like {\tt zadd}, the
implementation is optimised to assume that this
order is more likely to be used than the alternative.
That is, we should, for example, write

\begin{alltt}
   zadd(a, a, b);
\end{alltt}

\noindent
rather than

\begin{alltt}
   zadd(a, b, a);
\end{alltt}

\noindent
This assumption is not made for non-commutative
functions.

When writting your own functions, be aware,
input-parameters are generally not declared {\tt const}
in libzahl. Currently, some functions actually make
modifications (that do not affect the value) to
input-parameters.
