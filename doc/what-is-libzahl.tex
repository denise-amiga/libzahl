\chapter{What is libzahl?}
\label{chap:What is libzahl?}

In this chapter, it is discussed what libzahl is,
why it is called libzahl, why it exists, why
you should use it, what makes it different, and
what is its limitations.

\vspace{1cm}
\minitoc


\newpage
\section{The name and the what}
\label{sec:The name and the what}

In mathematics, the set of all integers is represented
by a bold uppercase `Z' ({\bf Z}), or sometimes
double-stroked (blackboard bold) ($\mathbb{Z}$). This symbol
is derived from the german word for integers: `Zahlen'
[\textprimstress{}tsa\textlengthmark{}l\textschwa{}n],
whose singular is `Zahl' [tsa\textlengthmark{}l]. libzahl
[l\textsecstress{}\textsci{}b\textprimstress{}tsa\textlengthmark{}l]
is a C library capable of representing very large integers,
limited by the memory address space and available memory.
Whilst this is almost none of the elements in {\bf Z},
it is substantially more than available using the intrinsic
integer types in C. libzahl of course also implements
functions for performing arithmetic operations over
integers represented using libzahl. Libraries such as
libzahl are called bigint libraries, big integer libraries,
multiple precision integer libraries, arbitrary precision
integer libraries,\footnote{`Multiple precision integer'
and `arbitrary precision integer' are misnomers, precision
is only relevant for floating-point numbers.} or bignum
libraries, or any of the previous with `number' substituted
for `integer'. Some libraries that refer to themselves as bignum
libraries or any of using the word `number' support other
number types than integers. libzahl only supports integers.


\newpage
\section{Why does it exist?}
\label{sec:Why does it exist?}

libzahl's main competitors are GNU MP (gmp),\footnote{GNU
Multiple Precision Arithmetic Library} LibTomMath (ltm),
TomsFastMath (tfm) and Hebimath. All of these have problems:

\begin{itemize}
\item
GNU MP is extremely bloated, can only be complied with
GCC, can only be dynamically linked by default, it is
required to opt in for static linking, and requires that
you use glibc unless another C standard library was used
when GNU MP was compiled. Additionally, whilst its
performance is generally good, it can still be improved.
Furthermore, GNU MP cannot be used for robust applications.

\item
LibTomMath is very slow, infact performance is not its
priority, rather its simplicit is the priority. Despite
this, it is not really that simple.

\item
TomsFastMath is slow, complicated, and is not a true
big integer library and is specifically targeted at
cryptography.
\end{itemize}

libzahl is developed under the suckless.org umbrella.
As such, it attempts to follow the suckless
philosophy.\footnote{\href{http://suckless.org/philosophy}
{http://suckless.org/philosophy}} libzahl is simple,
very fast, simple to use, and can be used in robust
applications. Currently however, it does not support
multithreading, but it has better support multiprocessing
and distributed computing than its competitor.

Lesser ``competitors'' to libzahl include Hebimath and
bsdnt.

\begin{itemize}
\item
Hebimath is far from stable, some fundamental functions
are not implemented and some functions are broken. The
author of libzahl thinks Hebimath is promising, but that
it could be better designed. Like libzahl, Hebimath aims
to follow the suckless philosophy.
\end{itemize}


\newpage
\section{How is it different?}
\label{sec:How is it different?}

All big number libraries have in common that both input
and output integers are parameters for the functions.
There are however two variants of this: input parameters
followed by output parameters, and output parameters
followed by input parameters. The former variant is the
conventional for C functions. The latter is more in style
with primitive operations, pseudo-code, mathematics, and
how it would look if the output was return. In libzahl,
the latter convention is used. That is, we write

\begin{alltt}
   zadd(sum, augend, addend);
\end{alltt}

\noindent
rather than

\begin{alltt}
   zadd(augend, addend, sum);
\end{alltt}

\noindent
This can be compared to

\vspace{1ex}
$sum \gets augend + addend$
\vspace{1ex}

\noindent
versus

\vspace{1ex}
$augend + addend \rightarrow sum$.
\vspace{1ex}

libzahl, GNU MP, and Hebimath use the output-first
convention. LibTomMath and TomsFastMath use the
input-first convention.

Unlike other bignum libraries, errors in libzahl are
caught using {\tt setjmp}. This ensure that it can be
used in robust applications, catching errors does not
become a mess, and it minimises the overhead of
catching errors. Errors are only checked when they can
occur, not also after each function-return.

Additionally, libzahl tries to keep the functions'
names simple and natural rather than techniqual or
mathematical. The names resemble those of the standard
integer operators. For example, the left-shift, right-shift
and truncation bit-operations in libzahl are called
{\tt zlsh}, {\tt zrsh} and {\tt ztrunc}, respectively.
In GNU MP, they are called {\tt mpz\_mul\_2exp},
{\tt mpz\_tdiv\_q\_2exp} and {\tt mpz\_tdiv\_r\_2exp}.
The need of complicated names are diminished by resisting
to implement all possible variants of each operations.
Variants of a function simply append a short description
of the difference in plain text. For example, a variant of
{\tt zadd} that makes the assumption that both operands
are non-negative (or if not so, calculates the sum of
their absolute values) is called {\tt zadd\_unsigned}.
If libzahl would have had floored and ceiled variants of
{\tt zdiv} (truncated division), they would have been
called {\tt zdiv\_floor} and {\tt zdiv\_ceiling}.
{\tt zdiv} and {\tt zmod} (modulus) are variants of
{\tt zdivmod} that throw away one of the outputs. These
names can be compared to GNU MP's variants of truncated
division: {\tt mpz\_tdiv\_q}, {\tt mpz\_tdiv\_r} and
{\tt mpz\_tdiv\_qr}.


\newpage
\section{Limitations}
\label{sec:Limitations}

libzahl is not recommended for cryptographic
applications, it is not mature enough, and its author
does not have the necessary expertise. And in
particular, it does not implement constant time
operations. Additionally, libzahl is not thread-safe.

libzahl is also only designed for POSIX systems.
It will probably run just fine on any modern
system. But it makes some assumption that POSIX
stipulates or are unpractical not to implement
for machines that should support POSIX (or even
support modern software):

\begin{itemize}
\item
Bytes are octets.

\item
There is an integer type that is 64-bits wide.
(The compiler needs to support it, but it is not
strictly necessary for it to be an CPU-intrinsic,
but that would be favourable for performance.)

\item
Two's complement is used.
(The compiler needs to support it, but it is not
strictly necessary for it to be an CPU-intrinsic,
but that would be favourable for performance.)
\end{itemize}

These limitations may be removed later. And there
is some code that does not make these assumptions
but acknowledge that it may be a case. On the other
hand, these limitations could be fixed, and agnostic
code could be rewritten to assume that these
restrictions are met.
