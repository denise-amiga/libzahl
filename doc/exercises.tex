\chapter{Exercises}
\label{chap:Exercises}

% TODO
% 
% All exercises should be group with a chapter
% 
% ▶   Recommended
% M   Matematically oriented
% HM  Higher matematical education required
% 
% 00  Immediate
% 10  Simple
% 20  Medium
% 30  Moderately hard
% 40  Term project
% 50  Research project
% 
% ⁺   High risk of undershoot difficulty


\begin{enumerate}[label=\textbf{\arabic*}.]


\item {[\textit{M10}]} \textbf{Convergence of the Lucas Number ratios}

Find an approximation for $\displaystyle{ \lim_{n \to \infty} \frac{L_{n + 1}}{L_n}}$,
where $L_n$ is the $n^{\text{th}}$ Lucas Number \psecref{sec:Lucas numbers}.

\( \displaystyle{
    L_n \stackrel{\text{\tiny{def}}}{\text{=}} \left \{ \begin{array}{ll}
      2 & \text{if} ~ n = 0 \\
      1 & \text{if} ~ n = 1 \\
      L_{n - 1} + L_{n + 1} & \text{otherwise}
    \end{array} \right .
}\)


\item {[\textit{M12${}^+$}]} \textbf{Factorisation of factorials}

Implement the function

\vspace{-1em}
\begin{alltt}
   void factor_fact(z_t n);
\end{alltt}
\vspace{-1em}

\noindent
which prints the prime factorisation of $n!$ (the $n^{\text{th}}$ factorial).
The function shall be efficient for all $n$ where all primes $p \le n$ can
be found efficiently. You can assume that $n \ge 2$. You should not evaluate $n!$.


\end{enumerate}



\chapter{Solutions}
\label{chap:Solutions}


\begin{enumerate}[label=\textbf{\arabic*}.]

\item \textbf{Convergence of the Lucas Number ratios}

It would be a mistake to use bignum, and bigint in particular,
to solve this problem. Good old mathematics is a much better solution.

$$ \lim_{n \to \infty} \frac{L_{n + 1}}{L_n} = \lim_{n \to \infty} \frac{L_{n}}{L_{n - 1}} = \lim_{n \to \infty} \frac{L_{n - 1}}{L_{n - 2}} $$

$$ \frac{L_{n}}{L_{n - 1}} = \frac{L_{n - 1}}{L_{n - 2}} $$

$$ \frac{L_{n - 1} + L_{n - 2}}{L_{n - 1}} = \frac{L_{n - 1}}{L_{n - 2}} $$

$$ 1 + \frac{L_{n - 2}}{L_{n - 1}} = \frac{L_{n - 1}}{L_{n - 2}} $$

$$ 1 + \varphi = \frac{1}{\varphi} $$

So the ratio tends toward the golden ration.


\item \textbf{Factorisation of factorials}

Base your implementation on

\( \displaystyle{
    n! = \prod_{p~\in~\textbf{P}}^{n} p^{k_p}, ~\text{where}~
    k_p = \sum_{a = 1}^{\lfloor \log_p n \rfloor} \lfloor np^{-a} \rfloor.
}\)

There is no need to calculate $\lfloor \log_p n \rfloor$,
you will see when $p^a > n$.


\end{enumerate}
